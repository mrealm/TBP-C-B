\chapter{Chapter Membership}
\section{Definition} The chapter membership consists of those persons so defined by C-VII,1 of the Constitution of the Association.

\section{Eligibility of New Members}\label{by:elig}
In addition to the requirements in \href{http://www.tbp.org/off/ConstBylaw.pdf}{C-II of the Constitution of the Association}, candidates must meet the following requirements:
\begin{enumsubsection}
\item{Residency} To be eligible for membership in the Chapter, an undergraduate student must have completed the equivalent of two full-time terms at the University of Michigan--Ann Arbor with at least one term in the College of Engineering, or be within one semester of graduation from the University of Michigan--Ann Arbor College of Engineering, or have been invited to join Tau Beta Pi at their previous school.
\item{Graduate Student Eligibility}\label{sec:grad_elig} To be eligible for election, graduate students must have additionally completed 11 credit hours of coursework as a graduate student. The determination of percent completion of coursework as required by C-II of the Constitution of the Association is to be made by a student's primary advisor or program coordinator. Graduate students who were eligible at the time of graduation from their undergraduate institution are eligible for election as alumni members.
\item{Eligible Undergraduate Curricula}\label{sec:ugrad_cur}  Undergraduates in the following curricula in the College of Engineering are eligible for membership in the chapter:

\let\oldenumi\labelenumii
\renewcommand{\labelenumii}{\arabic{enumii}.}
\begin{enumsubsubsection}

\itemnotoc Aerospace Engineering
\itemnotoc Biomedical Engineering
\itemnotoc Chemical Engineering
\itemnotoc Civil Engineering
\itemnotoc Civil and Environmental Engineering
\itemnotoc Climate and Space Sciences And Engineering
\itemnotoc Computer Engineering
\itemnotoc Computer Science
\itemnotoc Data Science
\itemnotoc Earth System Science and Engineering
\itemnotoc Electrical Engineering
\itemnotoc Engineering
\itemnotoc Engineering Physics
\itemnotoc Industrial and Operations Engineering
\itemnotoc Materials Science and Engineering
\itemnotoc Mechanical Engineering
\itemnotoc Naval Architecture and Marine Engineering
\itemnotoc Nuclear Engineering and Radiological Sciences

 \end{enumsubsubsection}
\let\labelenumii\oldenumi
\item{Eligible Graduate Curricula} Graduate students in any of the curricula listed in Bylaw~\ref{sec:ugrad_cur} or any of the following curricula are eligible for membership in the chapter:
\let\oldenumi\labelenumii
\renewcommand{\labelenumii}{\arabic{enumii}.}
\begin{enumsubsubsection}
\itemnotoc Automotive Engineering
\itemnotoc Computer Science and Engineering
\itemnotoc Construction Engineering and Management
\itemnotoc Electrical Engineering: Systems
\itemnotoc Energy Systems Engineering
\itemnotoc Environmental Engineering
\itemnotoc Financial Engineering
\itemnotoc Global Automotive and Manufacturing Engineering
\itemnotoc Integrated Microsystems
\itemnotoc Macromolecular Science and Engineering
\itemnotoc Manufacturing Engineering
\itemnotoc Pharmaceutical Engineering
\itemnotoc Plastics Engineering
\itemnotoc Space Engineering
\itemnotoc Structural Engineering
\end{enumsubsubsection}
\let\labelenumii\oldenumi
\end{enumsubsection}

\section{Process} The process for joining is referred to as the election process. The requirements and process for joining consist of those requirements and items outlined in C-III of the Association, and in the chapter Bylaws.

\section{Removal} The process for removing a member from membership is defined in C-I,5 of the the Constitution of the Association. % Added in October 2017 to comply with CCI's ridiculous requirements for constitutions. See Bylaws for MI-G process.

\section{Support} Upon joining the organization, all members agree not to undermine the purpose or mission of the Michigan Gamma chapter of Tau Beta Pi.
