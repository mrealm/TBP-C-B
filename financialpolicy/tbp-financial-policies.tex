\documentclass[bylaws,final,10pt,withoutoptional,withoutpreface,officerdoc]{../bylaws}
\usepackage[color]{changebar}
\usepackage{mathpazo}
\usepackage{xspace}
\usepackage{xcolor}
\organization{Tau Beta Pi -- Michigan Gamma Chapter}
\orglogo{../RotatedBentwWords}
\adoptiondate{3 December 2017}
\chapteramendmentdate{21 January 2018}
\appendixamendmentdate{-} 
\lastreviseddate{21 January 2018}
\officerdocumenttitle{Financial}
\begin{document}
\cbcolor{red}

\newcommand{\removed}[1]{\cbstart\removedfragile{#1}\cbend{}}
\newcommand{\removedfragile}[1]{{\color{red}{#1}}{}}
\newcommand{\added}[1]{\cbstart\addedfragile{#1}\cbend{}}
\newcommand{\addedfragile}[1]{{\color{green!50!black}{#1}}{}}
\newcommand{\changed}[2]{\added{#1}\removed{#2}}



\frontmatter
\maketitle
%
%
\tableofcontents\newpage
\mainmatter
\begin{optionalpart}
\part{Treasurer's Financial Policies}
\end{optionalpart}



\chapter{Chapter Budget}\label{sec:budget}
 The Chapter Constitution and Bylaws provide that the chapter will construct a budget. This section of the Treasurer's financial policy details the process and requirements to achieve this goal.

\section{General Policy} The members of the officer corps may petition the Treasurer for a budget to support their chapter related initiatives, responsibilities, and activities.

\section{Budget Authority} All members of the chapter's officer corps and chairs of standing committees are eligible to request a budget; hereafter {\it officers}. 

\section{Expense Cap} The budget proposed by the treasurer will not project to spend or encumber  more than 45\% of the prior fiscal year's gross income. The advisory board may disregard this limit by a $\frac{5}{7}$ majority. 

\section{Additional Budgets} Members of the Executive Committee may  authorize additional individuals within the chapter to request a budget subject to appeal to the full officer corps.
\section{Treasurer Discretion} The Treasurer may delegate certain budgeting authorities to any member or members of the officer corps.

\section{Request} Individuals seeking funding through the chapter will submit a written funding request to the Treasurer by a set deadline each semester. The treasurer will publish the funding form in advance of the deadline. The request document will group the requested funds into the pools listed in Policy \ref{sec:pools}

\begin{enumsubsection}
\itemnotoc The treasurer will meet electronically or in person with each officer requesting a budget.  
 \begin{enumsubsubsection}
   \itemnotoc The officer requesting a budget will present her/his ideas for the semester
   \itemnotoc The treasurer will review the prior semesters' related budgets. The requesting officer will identify which prior events and activities to continue.
    \itemnotoc The appropriate team lead and/or president will be invited to join the meeting.
 \end{enumsubsubsection}
\itemnotoc Each Team and Committee will meet to find budget efficiencies and to compose a team-wide budget.
 \begin{enumsubsubsection}
 	\itemnotoc Discretionary funds will be consolidated and placed under the direction of the team lead.
\end{enumsubsubsection}
\itemnotoc The treasurer will compile requests from officers and teams into a single budget proposal. 
 \begin{enumsubsubsection}
 	\itemnotoc The treasurer may exclude any  expense from his/her budget proposal with cause. The reason for the exclusion will be transmitted to the requesting officer, team, or committee. If they wish to appeal this decision, they may communicate their objection to the Advisory Board for consideration during the budget approval process.
	\itemnotoc The proposed budget will be presented to the executive committee for review.
	\itemnotoc The proposed budget will be presented to the Advisory Board at its first meeting of the semester for modification and / or approval.
\end{enumsubsubsection}

\end{enumsubsection}




\chapter{Budget Pools}\label{sec:pools}
The Chapter groups expenses into a set of pools. Individuals with budget authority may reallocate funds within a pool but not between pools without the written permission of the Treasurer.
\section{General} The General Pools cover the majority of chapter expenses. A request for funds in these pools requires a basic description of the proposed event or activity as well as a short explanation as to each expense's role in the proposed activity or event.
\begin{enumsubsection}
\itemnotoc Maintenance 
\itemnotoc Promotional
\itemnotoc Services
\itemnotoc Supplies
\itemnotoc Travel
\itemnotoc Entertainment
\itemnotoc Revenue
\itemnotoc Restricted Revenue
\end{enumsubsection}

\section{Special} A request for funds falling into a special pool require additional justification as compared to the General pools. Requests to special pools additionally require a justification as to each expense's relationship to the chapter and the Association's objectives.
\begin{enumsubsection}
\itemnotoc Awards
\itemnotoc Capital Goods
\itemnotoc Discretionary
\itemnotoc Food 
\end{enumsubsection}


\chapter{Spending Authority}\label{sec:authority}
Upon approval by the Advisory Board, officers can start to execute their budgets subject to the following rules and procedures.
\section{Budgeted Expenses} Any expense explicitly approved within the approved budget may be executed without additional approvals.
\section{Budget Reallocations} Any expense not explicitly approved within the approved budget but similar, may be executed by the officer with:
 \begin{enumsubsection}\label{sec:reallocations}
 	\itemnotoc No additional approvals if the expense is less than \$150. 
	\itemnotoc Approval of the Team Lead if the expense if the expense is greater than or equal to \$150 and less than \$300. The officer must notify the treasurer of the reallocation.
	\itemnotoc Approval of the Team Lead and the Treasurer if the expense exceeds \$300 and is less than \$1000.
	\itemnotoc Approval of the officer corps if the expense exceeds \$1000.
\end{enumsubsection}


\section{New Expenses} Any expense not explicitly approved within the approved budget and not similar to approved expenses, may be executed by the officer with:
 \begin{enumsubsection}\label{sec:newexpenses}
 	\itemnotoc Approval of the Treasurer if the expense is less than \$50.
 	\itemnotoc Approval of the Treasurer and Team Lead if the expense is greater than or equal to \$50 and less than \$100.
	\itemnotoc Approval of the Officer Corps with notification to the Advisory Board if the expense exceeds \$100.
\end{enumsubsection}

\section{Appeal} Declined requests for approval falling under Policies \ref{sec:reallocations} or \ref{sec:newexpenses} may be appealed to the officer corps. 

\section{Restrictions} The following class of expenses will not be approved for use in the chapter budget without approval of a majority vote of the Officer Corps.
\begin{enumsubsection}
\itemnotoc Parking
\itemnotoc Personal Fines
\end{enumsubsection}

\section{Misuse of Chapter Funds} In the event that a request is submitted for reimbursement of funds used inappropriately, the request will be reviewed by the Treasurer and will not be processed without the approval of $\frac{5}{7}$ of the advisory board. 

\chapter{Reimbursements}\label{sec:reimbursement}

\section{Process} Members must first receive approval for any expenditure to be reimbursed before a request is submitted. After approval has been granted, and the purchase made, the member will need to submit the following information to the Treasurer:
\begin{enumsubsection}
	\itemnotoc Description of event/purchase
	\itemnotoc The amount for TBP to reimburse
	\itemnotoc The name of the individual who needs to be reimbursed
	\itemnotoc The itemized receipts for the purchase with payment method
	\itemnotoc The address where the reimbursement check should be sent
\end{enumsubsection}

\section{Itemized Receipt} An itemized receipt will be required for reimbursement. In extreme cases, when an itemized receipt is unavailable for the purchase, the Treasurer can produce a memo for approval by the President to have the expense reimbursed. Proof of payment and further event details may be required; the Treasurer will be responsible for procuring this information when necessary. The signed memo will be submitted to Student Organization Account Services (SOAS) and, pending approval by SOAS, will be reimbursed.

\section{Services Rendered/Vendor Payments} These purchase requests apply to the chapter paying either a company or contractor directly. For this type of request, the member organizing the event will need to forward the invoice for the purchase and the contact information of the contractor/company directly to the treasurer. For contractors who are UM students or faculty, the member must also provide the contractor's UMID. {\it Non-UM contractors will be required to submit a W-9 if they have not been paid by the University in the past two years; this should be provided if possible in the initial request, or otherwise the process may be delayed.}

\section{Inventory of Supplies} The chapter officer corps will maintain an inventory spreadsheet of items that the chapter is currently in possession of. Items to be included in the inventory are: compostable materials for meetings, supplies for service events, DA/PA gifts, apparel items, K-12 event supplies, etc. Before purchasing for their event, project leaders and officers should refer to the inventory sheet to confirm that the purchase is required. If the needed supplies are already possessed by the chapter, the member should contact the treasurer and president to coordinate pick-up of supplies. If a purchase is still required, the members should follow the standard reimbursement procedure to complete the purchase. Any leftover supplies from the event should then be properly stored and recorded on the inventory sheet.



\end{document}
