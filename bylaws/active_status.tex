\section{Maintaining Active Status} To achieve active status, a Michigan Gamma Tau Beta Pi member must satisfy one of the following:
\begin{enumsubsection} %maybe change numbering type
\itemnotoc	Be an officer, Distinguished Active, or Prestigious Active,
\itemnotoc	Have initiated into the society in the previous semester,
\itemnotoc	Have returned to campus at the start of the current semester, and had been active prior to leaving, or
\itemnotoc	Achieve a sufficient level of activity in the society, as defined by the following requirements:
\begin{enumsubsubsection}
\itemnotoc	Attend, at a minimum, two meetings, one of which must be a voting meeting (2nd Actives, 3rd Actives or Officer Elections),
\itemnotoc	Complete two hours of TBP-sponsored service, and
\itemnotoc	Attend at least one social event.
\end{enumsubsubsection} 
\end{enumsubsection} 
In addition, any alumni member must request active status each semester, in accordance with \href{http://www.tbp.org/off/ConstBylaw.pdf}{C-VI of the Constitution of the Tau Beta Pi Association}.

\section{Tracking Status}  Active status is evaluated by the Membership Officer, and the designation is kept through the completion of the following semester.  If an inactive member satisfies the above requirement mid-semester, he/she can also be made active immediately upon request.  Additionally, active members who leave campus for one or more semesters will recover their status upon their return.
\section{Benefits of Active Status}  Active status may be designated on some chapter documents, and may be used in determining eligibility for certain chapter benefits at the discretion of the officer corps.  However, inactive members do not lose privileges (other than voting) relating to standard meetings and socials.