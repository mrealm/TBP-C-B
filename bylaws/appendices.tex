%%Officer Corps Changeable
\chapter{Undergraduate Electee Requirements}\label{sec:ugradreqs}
\section{Meetings} Each electee must attend all meetings as specified in Bylaw~\ref{sec:genmeetings}, except the general actives meeting and the election of candidates  meetings. A missed meeting may be made up with one extra hour of service that is approved by the Vice President. 
With the exception of Election of Officers meeting, a missed meeting may alternatively be made up by attending an extra social event approved by the Vice President. Given the typical length of the Officer Elections Meeting, electees who stay for the full duration will be awarded 2 service hours. At the two hour mark attendance will be retaken and those who remain will be awarded 1 service hour, those who stay until the end will be awarded 2.
\section{Character} Each electee will be assessed for exemplary character. This will consist of filling out a student information survey with questions focusing on exemplary character, and attending two interviews, not longer than thirty minutes each, each given by at least one active member.
\section{Dues} Each electee must pay the one-time-only membership dues as set by the Officer Corps. Should an electee be unable to pay the fee, he/she may submit a chapter fee waiver form and apply for a National Initiation-Fee Loan. The electee must provide sufficient evidence of financial need. The President, Vice President, and Treasurer will review the application to either deny or grant the waiver.
\section{Electee Exam} Each electee must complete an electee exam that is written by the Vice President with input from the officers. This exam will include a section about the history of Tau Beta Pi, a section about the chapter's officers, and a signature form, requiring at least two signatures from chapter officers, and at least five signatures from active Tau Beta Pi members who are not officers.
\section{Peer Interviews} Each electee must conduct interviews of six fellow Tau Bates, of which at least two must be active members. The interviews will consist of six creative questions. 
\section{Career Fair}\label{ugrad:CF} In the fall terms, each electee must complete two hours of service for Career Fair in jobs specified by the External Vice Presidents. 
\section{Tutoring} Each electee must complete at least one hour of university tutoring. The tutoring requirement may be satisfied either by flyering for tutoring or by serving as a tutor.
\section{K-12 Outreach} \label{ugrad:MindSET} Each electee must participate in at least one approved K-12 Outreach event for a minimum of 3 hours as determined by the K-12 Outreach Officers.
\section{Service} Each electee must complete at least twelve additional hours of service in the Fall, and fourteen additional hours in the Winter through at least two different service projects offered by the chapter. Any service completed in the events listed in \ref{ugrad:CF}--\ref{ugrad:MindSET} above the minimum may be used to satisfy this requirement. Electees may complete up to five service hours on their own if they provide adequate proof of their involvement. Service performed for the completion of requirements of another society will not be accepted. Extra service hours can be used to make up missed meetings.
\section{Socials and Professional Development} Each electee must attend at least two social or professional development events sponsored by Tau Beta Pi. Extra socials or professional development events can be used to make up missed meetings (except Officer Elections).
\section{Group Meetings} \label{group_meetings} Each electee must attend two group activities with at least half of their electee group present. One electee group activity must occur before the election of candidates, and the other before the third electees/actives meeting. This can be in the form of a social or service activity but does not have to be a Tau Beta Pi sponsored event. A missed group meeting may be made up with one extra hour of service or an extra social that is approved by the Vice President.
\section{Initiation} Each electee must attend initiation at the end of the term. This is an absolute requirement and cannot be made up. An electee who has completed all other requirements but misses initiation will not become a member of Tau Beta Pi. He/she may either attend the initiation of another chapter or wait until the following term to attend initiation.
\section{Timing and Deadlines}  The Vice President, with input from the Officer Corps, will establish appropriate deadlines for the requirements each semester. An electee may choose to count any activities completed after the final deadline but before the end of the term toward the following semester, or, if approved by the Vice President, the current term.

%%Officer Corps Changeable
\chapter{Graduate Electee Requirements}\label{sec:gradreqs}%%In much need of updating
\section{Meetings} Each electee must attend all meetings as specified in Bylaw~\ref{sec:genmeetings}, except the general actives meeting and the election of candidates  meetings. A missed meeting may be made up with one extra hour of service that is approved by the Grad Student Vice President. 
With the exception of Election of Officers meeting, a missed meeting may alternatively be made up by attending an extra social event approved by the Graduate Student Vice President. Given the typical length of the Officer Elections Meeting, electees who stay for the full duration will be awarded 2 service hours. At the two hour mark attendance will be retaken and those who remain will be awarded 1 service hour, those who stay until the end will be awarded 2.
\section{Character} Each electee will be assessed for exemplary character. This
will consist of filling out a student information survey with questions focusing on exemplary character.
\section{Service}	In addition to the requirements stated in Constitution Article IV.2.b, the electee must complete no less than ten service hours. 
\section{Interview}	The electee must undergo an interview by the Graduate Student Vice President and/or at least one active member. It is recommended that each interviewer be a graduate student. 
\section{Advisor Form}	Each electee must get the standard advisor recommendation form signed by their advisor or graduate program coordinator and turned into the Graduate Student Vice President by their interview.
\section{Dues}	Each electee must pay the one-time only membership dues as set by the Officer Corps. 
\section{Socials}	Each electee must attend at least two social  or professional development events sponsored by Tau Beta Pi.  At least one of these social events must be a graduate student social as designated by the Graduate Student Vice President.
\section{Initiation}	Each electee must attend initiation at the end of the term. This is an absolute requirement and cannot be made up. An electee who has completed all other requirements but who misses initiation may not become a member of Tau Beta Pi. He/she may either attend the initiation of another chapter or must wait until the following term to attend the initiation.  
\section{Timing and Deadlines}  The Graduate Student Vice President, with input from the Vice President, will establish appropriate deadlines for the graduate student requirements each semester. An electee may choose to count any activities completed after the final deadline but before the end of the term toward the following semester, or, if approved by the Graduate Student Vice President, the current term.


%%Officer Corps Changeable
\chapter{Distinguished Active Status Guidelines}\label{sec:DAstatus}
\section{Requirements} Members may achieve Distinguished Active status by completing at minimum the following requirements in the course of one term (Fall or Winter Terms):
\begin{enumsubsection}
\item{Leadership:} Serve as a group leader, project leader, committee chair,  officer/advisor, or achieve a sufficient level of involvement in a committee as defined in Appendix~\ref{sec:commPart}.
\item{Interviews:} Conduct one interview in the exemplary character assessment process of electees  Graduate students may fulfill this requirement by assisting in Graduate Student Interviews.
\item{Voting Meeting Attendance}\label{sec:DAvotingMeeting} Attend both election of candidate meetings and the election of officers meeting (Second and Third Actives and Elections). If missed, each of these meetings must be made up with an hour of service. Given the typical length of the Officer Elections Meeting, those who stay for the full duration will be awarded 2 service hours. At the two hour mark attendance will be retaken and those who remain will be awarded 1 service hour, those who stay until the end will be awarded 2.
\item{Meeting Attendance} Attend two general meetings in addition to the three listed in (\ref{sec:DAvotingMeeting}). These meetings, if missed, may be made up by either completing an extra hour of service or attending an extra social event.
\item{Service}\label{sec:DAservice} Complete at least eight hours of Tau Beta Pi service projects (project leaders may count only the participating hours of their project toward these hours). A single event cannot be used to simultaneously meet the leadership requirement and the entirety of the service hours requirement.
\item{Socials and Professional Development} \label{sec:DAsocial}Attend two social or professional development activities (social or professional development event leaders may count their event). In addition, electee group leaders may count additional electee group meetings (as defined in Appendix \ref{group_meetings}) as socials.
\item{Extra} Participate in either one more service hour or one more social event than required above in (\ref{sec:DAservice}) or (\ref{sec:DAsocial}).

\end{enumsubsection}
%\section{Summer DA Requirements:}
%\begin{enumsubsection}
%\item{Leadership:} Serve as a project leader, committee chair,  officer/advisor, or achieve a sufficient level of involvement in a committee as defined in Appendix~\ref{sec:commPart}.
%\item{Service}\label{sec:summerDAservice} Complete at least ten hours of Tau Beta Pi service projects (project leaders may count only the participating hours of their project toward these hours). A single event cannot be used to simultaneously meet the leadership requirement and the entirety of the service hours requirement.
%\item{Socials} \label{sec:summerDAsocial}Attend two social activities (social event leaders may count their event). 
%\item{Extra} Participate in at least 5 more hours than required above in (\ref{sec:summerDAservice}) or (\ref{sec:summerDAsocial}). Each of these hours can be earned by completing an hour of service or attending a social event.
%\end{enumsubsection}
\section{Attaining DA during Initiation Semester} An electee may count any hours beyond those needed for initiating toward achieving DA status. Requirements follow.
\begin{enumsubsection}
\item{Leadership:} Serve as a project leader, committee chair,  officer/advisor, or achieve a sufficient level of involvement in a committee as defined in Appendix~\ref{sec:commPart}. If an electee is unable to meet this requirement, an hour of service may be substituted.
\item{Service}\label{sec:electeeDAservice} Complete at least eight hours of Tau Beta Pi service projects (project leaders may count only the participating hours of their project toward these hours). 
\item{Socials} \label{sec:electeeDAsocial} Attend one social or professional development activity (social or professional development event leaders may count their event). Any electee group meetings beyond the required number can count towards this requirement.
\item{Extra} Participate in at least 3 more hours than required above in (\ref{sec:electeeDAservice}) or (\ref{sec:electeeDAsocial}). Each of these hours can be earned by completing an hour of service or attending a social event.
\end{enumsubsection}
\section{National Convention Distinguished Active Status} In addition, an active member may achieve Distinguished Active status by solely attending the Tau Beta Pi National Convention during that term, so long as he or she attends all business meetings and all other activities relevant to the chapter.
\section{Gift} At the end of each term, in appreciation for his/her contribution, each Distinguished Active will be given the following benefits according the number of terms of distinguished activity:
\begin{enumsubsection}
\item{First Term:} A Tau Beta Pi stole to be worn at graduation.
\item{Third Term:}  Invitation to attend the initiation banquet at no cost, where he/she will receive special recognition for their achievement and dedication. 
\item{Multiple Terms:} Additional benefits awarded for two or more  terms of Distinguished Active status will also be determined at the discretion of the Officer Corps. 
\end{enumsubsection}
\section{Timing and Deadlines} Any requirements completed following initiation but before the end of term may be counted toward the current term or the following full term. In addition, any requirements completed between the winter and fall terms may be counted toward the fall. 
\section{Substitution} Where circumstances merit, the Membership Officer may choose to allow any member to substitute one form of requirement for another, provided that the total number of hours completed is not diminished as a result of this.

%%Officer Corps Changeable
\chapter{Prestigious Active Status Guidelines}\label{sec:PAstatus}
\section{Requirements} Members may achieve Prestigious Active status by completing at minimum the following requirements in the course of one term:
\begin{enumsubsection}
\item{DA Status:} Achieving Distinguished Active status (as defined in Appendix~\ref{sec:DAstatus}.
\item{Total Involvement} Completing 32 total hours (including those from Distinguished Actives Requirements), defined as follows:
\begin{compactenum}[1.]
\itemnotoc Service hours and social or professional development credits count as 1 hour
\itemnotoc No more than 15 hours can be counted from a single event
\itemnotoc Only 8 total social or professional development credits  can be counted
\itemnotoc Hours must come from at least 3 different service events.
\end{compactenum}
\end{enumsubsection}
\section{Gift} At the end of each term, in appreciation for his/her significant contribution, each Prestigious Active will be given a fitting reward to be determined at the discretion of the Officer Corps, including invitation to attend the initiation banquet at no cost, where he/she will receive special recognition for his/her achievement and dedication.
\section{Timing and Deadlines} Any requirements completed following initiation but before the end of term may be counted toward the current term or the following full term. In addition, any requirements completed between the winter and fall terms may be counted toward fall. 

\section{Attaining PA during Initiation Semester} An electee may count any hours beyond those needed for initiating toward achieving PA status. Electee PA requirements are the same as those for an active except that the total number of hours required is 33 (34 if no leadership credit is earned). Additionally, up to 10 social credits may be counted toward this number.
\section{Substitution} Where circumstances merit, the Membership Officer may choose to allow any member to substitute one form of requirement for another, provided that the total number of hours completed is not diminished as a result of this.
%\section{Attaining PA during Summer} A member achieve PA status during the summer term. Summer PA requirements are the same as those for a full term except that the total number of hours required is 38. Additionally, up to 14 social credits may be counted toward this number.

%%NOT Officer Corps Changeable
\chapter{Officer Requirements and Descriptions}\label{sec:officerreq}
Descriptions for each officer position follow below. In addition to the specific duties of each office an officer is expected to complete at least four service hours, attend at least two Tau Beta Pi socials as well as both the initiation and the banquet.  These requirements are necessary for obtaining Distinguished Active status but are not sufficient.  To become a Distinguished Active an officer must also satisfy the requirements in Appendix~\ref{sec:DAstatus}. Appendix~\ref{sec:ExecComm}--\ref{sec:ElecteeTeam} list only permanent officer members of each Team. Ad hoc officers and their team affiliation are listed in Appendix~\ref{sec:AdHocOfficers}.

\section{Executive Committee}\label{sec:ExecComm}
These officers serve as members of the Executive Committee.
\begin{enumsubsection}
\item{President} The President is the official representative, spokesperson, chief executive officer and chief operating officer of Tau Beta Pi Michigan Gamma. He/she, with the assistance of Team Leads, ensures that each officer and chair is provided with a written list of all duties for which he/she is responsible and sees that they are fulfilled.  He/she must also cause to be obtained a list of eligible candidates for election prior to the beginning of each term. He/she must also prepare a meeting schedule at the beginning of the term and is responsible for scheduling rooms for the general meetings. The President serves as the lead for the Executive Committee. 

\item{Vice President} The Vice President oversees all electee activities and business. He/she is responsible for ensuring that all electees complete the election requirements prior to initiating. He/she also assists the President with leadership and planning duties, and is the alternative representative of Tau Beta Pi in lieu of the President. He/she  is also responsible for conducting the exemplary character process of prospective undergraduate members as specified in Appendix \ref{sec:ugradreqs}. The Vice President additionally serves as the lead for the Electee and Membership Team.

\item{Secretary} The Secretary is responsible for completing all paperwork and ensuring that it is submitted to the National Office on time, including but not limited to the Chapter Survey. He/she also serves as the main contact between the chapter and National Office. Additionally, the Secretary must fulfill the duties of the Cataloguer as specified in the National Bylaws 5.03. The Secretary holds the responsibilities of the Recording Secretary and is in charge of taking minutes at the meeting and making them available to the active membership, and of ensuring a timely transition of website permissions.

\item{Treasurer} The Treasurer is responsible for sales, dues, chapter fundraising activities, budgeting, and reporting of the chapter's finances. He/she is a signer in the chapter checking account and must review the chapter finances periodically. Additionally, he/she files annually (by May 15) IRS Form 990, 
 ``Return of Organization Exempt from Income Tax".  He/she serves as a non-voting member of the Advisory Board described in Bylaw~\ref{sec:advbrd}. The treasurer is additionally in charge of organizing, with assistance from the rest of the Executive Committee, the underclassmen mixer.

\item{Graduate Student Vice President} The Graduate Student Vice President is responsible for creating and tracking the electee progress of all graduate students and alumni candidates. He/She coordinates with the Vice President as needed. He/She sets up social events for graduate students and activities with other graduate student organizations. He/She conducts all meetings with graduate students and graduate student electees when necessary. He/she serves as a non-voting member of the Advisory Board described in Bylaw~\ref{sec:advbrd}.

\end{enumsubsection}

\section{Professional Development Committee}\label{sec:PDTeam}
These officers serve as members of the Professional Development Committee.
\begin{enumsubsection}
\item{Corporate Relations Officer} The Corporate Relations Officer serves as the Chair for the Professional Development Committee and is  responsible for the relations of the chapter to corporate representatives. He/she is in charge of planning the topics for the MLK Luncheons. He/she is also  in charge of scheduling speakers for the meetings and luncheons. 
\item{External Vice Presidents} The External Vice Presidents are responsible for the chapter's entire involvement with the Career Fair and the College of Engineering Honors Brunch. They chair the Tau Beta Pi awards committee. In the winter, they are in charge of planning the Career Fair and organizing the Honors Brunch. In the fall, they are in charge of running the Career Fair. They also serve as ex officio members of the Professional Development Committee, though they are overseen by the President.
\end{enumsubsection}

\section{Events Team}
These officers serve as members of the Events Team.
\begin{enumsubsection}
\item{Service Coordinator} The Service Coordinator oversees all service projects and must ensure that all project leaders complete their projects on time. He/she is also responsible for informing the electees and actives of all the projects and providing them with opportunities to sign up for the projects of their choice. Additionally, he/she will plan and coordinate new service projects and assist with the preparation of the Chapter Survey. The Service Coordinator serves as the lead for the Events Team.
\item{ K-12 Outreach Officers} The K-12 Outreach Officers seek and coordinate opportunities for engineering outreach within the community.  They serve as the liaisons for Chapter-sponsored K-12 outreach programs within both the Chapter and the community. This includes the chapter-run MindSET program. 

\item{The Campus Outreach Officer} The Campus Outreach Officer seeks and coordinates opportunities for outreach to College of Engineering students, including, but not limited to, coordinating a tutoring program and promoting outreach programs to the engineering campus. 

\item{Activities Officer} The Activities Officer oversees all chapter social events, including intersociety events, and ensures that any social project leaders complete their events on time. He/she will make arrangements for the initiation banquet. 
\end{enumsubsection}
% He/she is responsible for providing food and beverages at all general body meetings specified in Bylaw III.
\section{The Chapter Team}
These officers serve as members of the Chapter Team.
\begin{enumsubsection}
\item{Chapter Development Officer} The Chapter Development Officer is responsible for working with the Executive Committee to investigate and carry out ways of pursuing new opportunities or improvements for the chapter. The Chapter Development officer additionally is responsible for the planning and execution of New Initiatives meetings each semester, which are described in Bylaw~\ref{sec:NImeetings}. The Chapter Development Officer serves as the lead for the Chapter Team.

\item{Historian} The Historian is responsible for the generation and distribution of all chapter publications, including the chapter newsletter, which is named ``The Cornerstone", the Alumni Newsletter, and any website publications deemed necessary. The Historian is in charge of picture taking for the Chapter Survey and maintains chapter records, including member demographics, as needed. 

\item{Publicity Officer} The Publicity Officer is responsible for all internal and external chapter publicity, including the weekly announcements, social media presence (e.g. Facebook, Twitter), college-wide announcements, flyer generation, and any other publicity deemed necessary.

\item{Membership Officer} The Membership Officer is responsible for keeping track of all MI-$\Gamma$ members after they have been initiated. This includes determining who achieves active,  distinguished active, and prestigious active  status ,tracking meeting attendance, maintaining the email lists, and handling all alumni relations, except the publication of the Alumni Newsletter.  He/she is also responsible for providing food and beverages at all general body meetings specified in Bylaw~\ref{sec:genmeetings}.
\end{enumsubsection}
%
%\section{The Social Team}
%These officers serve as members of the Social Team.
%\begin{enumsubsection}
%\item{Activities Officer} The Activities Officer makes arrangements for all chapter social events. He/she is responsible for providing food and beverages at all general body meetings specified in Bylaw III. He/she will make arrangements for the initiation banquet. The Activities Officer serves as the lead for the Social Team.
%
%\item{Intersociety Officer} The Intersociety Officer is responsible for all intersociety activities, including but not limited to intramural sports and intersociety socials.
%\end{enumsubsection}
%
%\section{The Electee and Membership Team}\label{sec:ElecteeTeam}
%These officers serve as members of the Electee and Membership Team. The Vice President is the lead of this team, but is listed under Executive Committee in Appendix~\ref{sec:ExecComm}.
%\begin{enumsubsection}
%
%\end{enumsubsection}%vspace{.2in}

%%Officer Corps Changeable
\section{Ad Hoc Officers}\label{sec:AdHocOfficers} The ad hoc officers and their descriptions follow:\\
%\begin{enumsubsection}
%%\item{Summer Committee Chair---Expires August 30, 2014} \emph{Active May 1--August 30}	This position will serve as acting president during its term. It should be filled by the Fall President, if possible. Otherwise it should be filled by a former president or advisor. He/she will coordinate the local activities of the chapter during the summer months, head the Summer Committee, and will designate any needed vice-chairs.
%%\item{District 7 Conference Chair--Expires April 2013} shall coordinate the planning and running of the Tau Beta Pi District 7 Conference during any academic year in which MI-Gamma is hosting. He/she will additionally chair the District Conference Committee, which will assist in his/her duties. In years when MI-Gamma is not hosting a Disctrict 7 Conference, this position shall be left vacant. 
%\end{enumsubsection}

%%NOT Officer Corps Changeable
\chapter{Committees}\label{sec:committees}%Flesh this out with a section for the standing and a section for the ad hoc
\section{Committee Participation}\label{sec:commPart} At the beginning of each term, the Executive Committee, with input from the respective committee chairs, will establish guidelines for levels of involvement for each committee, including a minimum level of involvement necessary for gaining a leadership credit.
\section{Standing Committees}\label{sec:standingCommittees} The standing committees and their descriptions are as follows:
\begin{enumsubsection}
	\item{Professional Development Committee} The Professional Development Committee will consist of the Professional Development Team, and any other chapter members as determined by the Corporate Relations Officer, who chairs the committee. The committee is responsible for hosting Engineering Futures sessions, information sessions with companies, MLK Luncheons, and other professional development events.
	\item{Website Committee} The Website Committee is chaired by the Website Chair, includes members as determined by the Website Chair, and is responsible for maintaining of the chapter website and any supporting technology. Administrator privileges for the chapter website are only to be granted to officers, advisors, or committee members.
	\item{Book Swap Committee} The Book Swap Committee is chaired by the Service Coordinator, who may appoint co-chairs as appropriate. The Committee is responsible for the planning and execution of the semesterly Book Swap.
	\item{Group Leaders Committee} The Group Leaders Committee is chaired by the Vice President, who will select members as appropriate. The committee is composed of all electee group leaders and is responsible for coordinating the electee group aspect of the election process.
\end{enumsubsection}

%%Officer Corps Changeable
\section{Ad Hoc Committees}\label{sec:AdHocCommittees} The ad hoc committees and their descriptions follow:\\
\begin{enumsubsection}
\item{K-12 Committee---Expires January 26, 2016}  This committee will be co-chaired by the K-12 Outreach Officers and will serve to assist them in coordinating the K-12 Outreach efforts of the chapter, including but not limited to the MindSET program.
%\begin{enumsubsubsection}
%	\item{Summer Treasurer} The summer treasurer will be responsible for filing and issuing reimbursements during the summer months and for keeping the Treasurer informed of the financial activity of the chapter.. Where possible this should be filled by the current or a former treasurer, otherwise it should be filled by someone with experience as an authorized signer. The Summer Treasurer should be designated as the fourth authorized signer for the SOAS account.
%	\item{Summer Service Coordinator} The summer service coordinator will be responsible for coordinating service projects over the summer, and for ensuring that the appropriate Project Reports are completed.
%	\item{Summer Social Chair} The summer social chair will coordinate socials over the summer months. This should be filled by the current or a former Graduate Student Coordinator when possible.
%	\item{Summer Secretary} During the summer months, the summer secretary is responsible for taking minutes at committee meetings, determining who achieves DA/PA status, and sending out announcements to members as needed.
%\end{enumsubsubsection}
%%\item{District 7 Conference Chair--Expires April 2013} shall coordinate the planning and running of the Tau Beta Pi District 7 Conference during any academic year in which MI-Gamma is hosting. He/she will additionally chair the District Conference Committee, which will assist in his/her duties. In years when MI-Gamma is not hosting a Disctrict 7 Conference, this position shall be left vacant. 
\end{enumsubsection}

\chapter{Chairs}\label{sec:Chairs} 
\section{Current Chair Positions} The Chairs and their descriptions follow:\\
\begin{enumsubsection}
\item{Website Chair} The Website Chair is responsible for updating and maintaining the chapter website, and serves as the chair of the website committee. He/She serves on the Chapter Team.
\item{IM Sports Chair} The IM Sports Chair is responsible for the chapter's participation in Intramural Sports and by default serves as the captain of any such teams. He/She serves on the Events Team.
\item{Outreach Chair} The Outreach Chair is responsible for assisting the
campus outreach officer in the outreach activities and events of the chapter. This chair serves on the Events Team.
\item{Apparel Chair} The Apparel Chair is responsible for the design and acquisition of any MI-G branded apparel that the chapter may desire. This chair serves on the Chapter Team.
\item{Alumni Relations Chair} The Alumni Relations Chair is responsible for developing and maintaining the chapter's alumni relations initiatives. The chair will work with the Secretary and membership officer to maintain an up-to-date MI-G alumni contact roster. He/She serves on the Chapter Team.
\item{Graduate Student Activities Chair} The Graduate Student Activities Chair(s) will work with the graduate student vice president to help plan and execute events for graduate student actives and candidates. Further, the chair(s) will support the graduate student vice president in the execution of his/her duties. The chair(s) will be members of the Events Team but shall collaborate with the Graduate Student Vice President.
%\item{New Initiatives Chair} The New Initiatives Chair is responsible for running the chapter's New Initiatives Meetings, working with the Chapter Development Officer to set the agenda, and for providing such meeting food as is deemed appropriate. The New Initiatives Chair should be an officer or advisor and will additionally serve on the Chapter Team.
\end{enumsubsection}