\section{Chapter Officers} The officers of this chapter are  President, Vice President, Corresponding Secretary/Recording Secretary/Cataloguer 
(hereafter referred to as Secretary), Treasurer, two External Vice Presidents,
 Service Coordinator, Professional Development Officer, Activities Officer,  Graduate Student Coordinator, Chapter Development Officer, two K-12 Outreach
 Officers, Campus Outreach Officer, Membership Officer, 
Historian, Publicity Officer, and all ad hoc officers listed in Appendix~\ref{sec:AdHocOfficers}.  All officers must be  members. In addition to the duties specified in the \href{http://www.tbp.org/off/ConstBylaw.pdf}{Tau Beta Pi Association Bylaws 5.03}, the duties described in Appendix \ref{sec:officerreq} are required of the officers.

\section{Advisors}\label{sec:advisors} The chapter must maintain a minimum of four advisors, as stipulated in \href{http://www.tbp.org/off/ConstBylaw.pdf}{C-VI,7} of the Constitution of the Tau Beta Pi Association. An advisor must be an initiated member of Tau Beta Pi. Additionally, advisors must be a faculty member, %UM??
a national official, %%District Director count??
an alumni of MI-$\Gamma$ with appropriate chapter experience, a graduate student with appropriate chapter experience, or a former chapter president. One of the the advisors must serve as the chapter's Chief Advisor. This advisor will be selected by the advisors biannually for a two year term. The advisors are responsible for ensuring chapter continuity.

\section{Advisory Board}\label{sec:advbrd} The Advisory Board is composed of the Executive Committee and the Chapter Advisors.  The Treasurer and Graduate Student Vice President serve as non-voting members of the Advisory Board. The voting advisors for the Advisory Board are the Chief Advisor and the three most senior advisors.
\begin{enumsubsection}
\itemnotoc Seniority is measured as time spent in the current advisor term.
\itemnotoc In the event of a tie affecting determination of voting advisors, the President will determine the voting advisor(s) from among those tied by a suitably random method before the first contested vote of the semester.
\end{enumsubsection}

\section{Officer Corps}\label{sec:officercorps} 
\begin{enumsubsection}
\item{Membership and Responsibilities} The Officer Corps consists of the officers and advisors of the Chapter, and any other persons deemed necessary by the officers. The officer corps:
\begin{enumsubsubsection}
\item*{Initiation Dues} Sets the level of initiation dues.
\item*{Fund Administration} Administers funds available for Tau Beta Pi scholarships according to procedures established in writing by the officer corps, unless a different method of administration is specified for the funds.
\item*{Creation of Ad Hoc Officer Positions}\label{sec:AdHocCreation} May, with Advisory Board approval, create ad hoc officer positions. These positions:
\begin{enumerate}
	\itemnotoc Must be listed in Appendix~\ref{sec:AdHocOfficers}.
	\itemnotoc May be placed on a new or existing Team (see Section~\ref{sec:OfficerTeams}).
	\itemnotoc May only last for two academic terms, after which time they must be approved by the general body. This can be either as a permanent officer position, or as an extension of the ad hoc position. An extension of the ad hoc position requires a simple majority vote of the general body, and may not be for longer than another two terms. An officer position may not exist in an ad hoc state for more than four consecutive terms. Any ad hoc position that has existed for four consecutive terms cannot be recreated as ad hoc without a gap of at least two terms.
\end{enumerate}
\item*{Creation of Chair Positions}\label{sec:ChairCreation} May create Chair positions as needed. These positions:
\begin{enumerate}
	\itemnotoc Must be listed in Appendix~\ref{sec:Chairs}.
	\itemnotoc May be placed on an Officer Team (see Section~\ref{sec:OfficerTeams}).
\end{enumerate}
 \end{enumsubsubsection}
%Fill in these descriptions
\item{Officer Teams}\label{sec:OfficerTeams} The officer corps is organized into teams. These teams are the Executive Committee, Professional Development Team, Events Team, and Chapter Team. The lead and membership of each team is defined in Appendix~\ref{sec:officerreq}.
\begin{enumsubsubsection}
\item{Team Leads} Team Leads are responsible for the oversight and coordination of the team and for ensuring that team members tasks are completed. Team members retain responsibility for how their tasks should be completed. Team Leads are additionally responsible for serving as mentors for their team members when necessary.
\item{Purpose} Teams are intended to assist the President in overseeing the officers, to create additional opportunities for leadership growth, and to help facilitate mentoring of newer officers. They are not intended to segment the decision making of the officer corps, and as such most communication and discussion should happen amongst the entire officer corps.
\item{Executive Committee} In addition to the individual responsibilities of the constituent officers, the Executive Committee is collectively responsible for setting the semester calendar and for designating the Convention Delegate (fall term).
\end{enumsubsubsection}
\end{enumsubsection}
%\subsection{
% 1) define officer corps and responsibilities 2) officer teams - subsubsection for each team

\section{Committees} % 1) standing committees 2) temporary committees
\begin{enumsubsection}
\item{Purpose} Committees are intended to facilitate additional chapter activity and provide members additional involvement opportunities.  Committee function and schedule will be determined by the chair, with input from the committee members. Committees are distinct from Teams in that Teams are comprised of officers with similar but distinct responsibilities, while committees are open to any member, per their selection procedures, and are, in general,  geared toward a more focused task.
\item{Standing Committees} The standing committees are the Professional Development Committee, Website Committee, Book Swap Committee, and Group Leaders Committee. The membership and duties of each are listed in Appendix~\ref{sec:standingCommittees}.
\item{Ad Hoc Committees}\label{sec:AdHocCommitteeCreation} In addition to the standing committees, the officer corps, with Advisory Board approval, may create ad hoc committees. These committees:
\begin{enumsubsubsection}
\item*{Enumeration} Must be listed in Appendix~\ref{sec:AdHocCommittees}.
\item*{Chair} May be chaired by an officer or another active member, as selected by the officer corps. 
\item*{Membership} Will be composed of  members on a volunteer basis. (In the event of detrimental participation and at the recommendation of the committee chair, the officer corps may remove a member of the committee Members so removed may make an appeal to the Advisory Board, who may reverse the removal by a \nicefrac{5}{7} vote.)
\item*{Duration} May exist for a maximum of two years, after which time they must be approved by the general body. This can be either as a permanent committee, or as an extension of the ad hoc committee. An extension of the ad hoc committee requires a simple majority vote of the body, and may not be for longer than another year. A committee may not exist as ad hoc for more than three years. Any ad hoc committee that has existed for three years cannot be recreated as ad hoc without a gap of at least one year.
\end{enumsubsubsection}
\end{enumsubsection}
\section{Chairs}
\begin{enumsubsection}
\item{Purpose} Chair positions are intended to be single-purpose leadership roles within the chapter. They are intended to facilitate additional leadership opportunity within the organization, as well as to allow smaller and/or more specific tasks to be carried out by someone not necessarily an officer.
\item{Creation} Chair positions may be created by the officer corps or the general membership at any time by a simple majority vote at any time. Chair positions may exist for any length of time, though chairs should be appointed at least semesterly.
\item{Dissolution and Removal} Chair positions may be removed at any time by a 2/3 vote of the officers. Additionally, persons serving as chairs may be removed from their position by a majority vote of the officers.
\end{enumsubsection}

\section{Terms of Office} The officers of this chapter hold office for one semester except for the External Vice Presidents, 
one K-12 Outreach Officer, and Treasurer, whose terms are one calendar year, and the Secretary, the remaining K-12 Outreach 
Officer, and Professional Development Officer, whose term is one academic year. Ad hoc officer positions specified in Appendix~\ref{sec:AdHocOfficers}. will have terms of one semester, unless otherwise specified. Advisor terms are decided as part of the advisor election procedure, as described in Bylaw~\ref{sec:advisorLength}.
