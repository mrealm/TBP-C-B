%This is a "non-binding" intro to the bylaws that is intended as a transition material for the advisor managing them.
\begin{preface}
\chapter*{Preface}
Greetings Future Keeper of the Bylaws,\\
Oh where are my manners? First,
\begin{center}
\includegraphics[width=3in]{dont_panic}
\end{center}
Ok, now that that is taken care of, welcome to the Tau Beta Pi--Michigan Gamma Chapter Bylaws. It's really a wonderful place, and the fact that you are reading this means that you were forced, err, told, err selected, yes selected, to be in change of the maintaining of the chapter bylaws. Most of the time, it really shouldn't be anything substantial. They're laid out reasonably well, but it's important to be acquainted with what is where and why, so that you can make sure to propose changes to them as needed, and so that updating them won't be too much of a hassle. The table of contents which follows should be a good overview there. But here I'll try to go into some of the why with the bylaws.

Bylaw~\ref{sec:gen}. This is the boilerplate opening. Nothing too fancy here just denoting the succession of authority and deference to Nationals and to the University/College. If we at some point split to a separate Constitution \& Bylaws model we would denote the precedence of the Constitution here.

Bylaw~\ref{sec:gov}. This is all about chapter government. Officers, advisors, committees, teams---really everything except, I think, project leaders found its way into this one. If you were to ever add to the governing structure or formalize any aspects related to leadership and governance, this is the place to do it. In general, the guiding principle for naming officers has been to pick position titles which are semantically meaningful, without being restrictive, cumbersome, or unfathomable to corporate representatives. So for instance, ``Tutoring" was too restrictive and was changed to ``Campus Outreach". ``Honors Brunch and Career Fair Co-Chair" was too unwieldy and not meaningful to company reps so it was changed to ``External Vice President".  ``Intersociety" is a little more descriptive than other options like ``Social" or ``Events". You get the idea. I suppose a little history is also in order here. There were a number of substantive changes to this section in March 2013, which stemmed from an initiative to make the administrative load of the President more reasonable. At the time, the President was responsible for directly keeping tabs on all 21 other officers, virtually unassisted. As was recounted by any who served in the office during that time, this was completely unreasonable, and from this was born the concept of officer Teams. They were intended to divvy up some of the oversight to help keep the President sane---and to provide opportunities both for mentoring of newer officers by older officers and for leadership growth and ``stepping stones" for returning officers. Committees and ad hoc officers  were also added at that time to help add some flexibility to the officer corps and allow the chapter to adapt to new and changing opportunities---and to expand the chapter offerings to members and the College as a whole. Committees also would allow additional involvement opportunities for members who weren't officers to be a part of.

Bylaw~\ref{sec:oficElec}. This is everything about the election procedures that we specify for the chapter. Procedure, order, nominations, speeches, discussion, who can speak, etc. Anything not covered here is left to Robert's Rules to sort out. 

Bylaw~\ref{sec:advElec}. This is similar to the Bylaw on officer election, but notes the differences for election of Advisors. This was added in March 2013 to formalize the process for Advisors, which had been fairly hand-wavey until that point.

Bylaw~\ref{sec:appointmentofficers}. This covers which officers are appointed rather than elected, and the process to follow for each. It was first implemented in April 2012, when the District 7 Conference Chair was added. At the time it was written in a way that was very much intended for it to be a this-position-only Bylaw. That changed in October 2012 when External Vice President was moved to an application/interview system, to be able to better assess the 2 people's ability to work together, and to better mesh with SWE's selection of the other 2 chairs. In March 2013, this  was generalized to allow for convenient description of now several positions by application (EVP, Website, ad hoc positions).

Bylaw~\ref{sec:meetings}. This describes regularly held meetings. NI was added to this list in March 2013 to shore up its importance.

Bylaw~\ref{by:elig}. Title sums it up. New member eligibility...probably shouldn't change this much.

Bylaw~\ref{sec:newElec}. Goes along with the meetings and the appendices to lay out the process for joining.

Bylaw~\ref{sec:records}. Describes records and transitions. Not much done here. Since you're in charge of the bylaws you're likely in charge of transition material too. If not make sure that this is still being handled reasonably.

Bylaw~\ref{sec:activeStatus}. Describes what it takes to be minimally active. As of this writing this was not really enforced at all, but there are criteria set out so it could be.

Bylaw~\ref{sec:amendment}. Enactment and amendment of these bylaws hasn't really been touched much except for the part on appendices. If you add an appendix that is to be modifiable by the officer corps you need to note that hear.
%Continue noting the appendices

Appendix~\ref{sec:ugradreqs} Undergrad  Initiation Requirements. Have the VP update this if needed. Instruct them to read Bylaw~\ref{sec:newElec} and to not assume that anything has been previously defined in the process apart from that Bylaw. Modifiable by the Officer Corps.

Appendix~\ref{sec:gradreqs} Grad  Initiation Requirements. These are usually also applied to grad students considered as alumni. Have the Grad Coordinator update this if needed. Instruct them to read Bylaw~\ref{sec:newElec} and to not assume that anything has been previously defined in the process apart from that Bylaw. Modifiable by the Officer Corps.

Appendix~\ref{sec:DAstatus} DA status requirements and rewards. This also introduces the concept of a leadership credit. Modifiable by the Officer Corps.

Appendix~\ref{sec:PAstatus} PA status requirements and rewards. This inherits stuff from DA. Modifiable by the Officer Corps.

Appendix~\ref{sec:officerreq} Defines the responsibilities of each officer and Team, as well as the Team's membership. The part on ad hoc officers is modifiable by the Officer Corps and lists all of the existing ad hoc officers.

Appendix~\ref{sec:committees} Defines the responsibilities of the standing committees, and the existence and responsibilities of the ad hoc committees. The section on ad hoc committees is modifiable by the Officer Corps.

With that you should be set.\\

 All the best,

\includegraphics{ElectronicSignature}

Mike Hand

Chapter Advisor, Keeper of the Bylaws, and Official Lore Master, 2013-

\newpage
\end{preface}